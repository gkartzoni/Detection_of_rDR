% Ο πιο απλός τρόπος για τις αναφορές είναι ο παρακάτω όπου σε κάθε αναφορά ανατίθεται μία λέξη κλειδί για να ααντισοιχίσουμε το άρθρο μέσα στο 
% κείμενο και έπειτα ο τίτλος του άρθρου και οι συγγραφείς κ.λ.π. Η μορφή είναι η εξής:
% \bibitem{<λέξη-κλειδί>} Στοιχεία άρθρου ή βιβλίου

% Για να βρίσκετε βιβλία και άρθρα στην μορφή που θέλετε μπορείτε να αναζητήσετε τα άρθρα στην ιστοσελίδα citethisforme.com . Ενδείκνυται το
% πρότυπο της ΙΕΕΕ 

\begin{thebibliography}{99}
\addcontentsline{toc}{chapter}{Βιβλιογραφία}

\nocite{*}
\bibitem{Gargeya} Gargeya, Rishab, and Theodore Leng. “Automated Identification of Diabetic Retinopathy Using Deep Learning.” Ophthalmology, vol. 124, no. 7, 2017, pp. 962–969., doi:10.1016/j.ophtha.2017.02.008.

\bibitem{Acharya} Acharya, U R, et al. “Computer-Based Detection of Diabetes Retinopathy Stages Using Digital Fundus Images.” Proceedings of the Institution of Mechanical Engineers, Part H: Journal of Engineering in Medicine, vol. 223, no. 5, 2009, pp. 545–553., doi:10.1243/09544119jeim486.

\bibitem{Beagley} Beagley, Jessica, et al. “Global Estimates of Undiagnosed Diabetes in Adults.” Diabetes Research and Clinical Practice, vol. 103, no. 2, 2014, pp. 150–160., doi:10.1016/j.diabres.2013.11.001.

\bibitem{Ciulla} Ciulla, T. A., et al. “Diabetic Retinopathy and Diabetic Macular Edema: Pathophysiology, Screening, and Novel Therapies.” Diabetes Care, vol. 26, no. 9, 2003, pp. 2653–2664., doi:10.2337/diacare.26.9.2653.

\bibitem{Congdon} Congdon, Nathan G. “Important Causes of Visual Impairment in the World Today.” Jama, vol. 290, no. 15, 2003, p. 2057., doi:10.1001/jama.290.15.2057.

\bibitem{Giraddi} Giraddi, Shantala, et al. “Identifying Abnormalities in the Retinal Images Using SVM Classifiers.” International Journal of Computer Applications, vol. 111, no. 6, 2015, pp. 5–8., doi:10.5120/19540-9686.

\bibitem{Guariguata} Guariguata, L., et al. “Global Estimates of Diabetes Prevalence for 2013 and Projections for 2035.” Diabetes Research and Clinical Practice, vol. 103, no. 2, 2014, pp. 137–149., doi:10.1016/j.diabres.2013.11.002.

\bibitem{Gulshan} Gulshan, Varun, et al. “Development and Validation of a Deep Learning Algorithm for Detection of Diabetic Retinopathy in Retinal Fundus Photographs.” Jama, vol. 316, no. 22, 2016, p. 2402., doi:10.1001/jama.2016.17216.

\bibitem{Lin} Lin, Gen-Min, et al. “Transforming Retinal Photographs to Entropy Images in Deep Learning to Improve Automated Detection for Diabetic Retinopathy.” Journal of Ophthalmology, vol. 2018, 2018, pp. 1–6., doi:10.1155/2018/2159702.

\bibitem{Nayak} Nayak, Jagadish, et al. “Automated Identification of Diabetic Retinopathy Stages Using Digital Fundus Images.” Journal of Medical Systems, vol. 32, no. 2, 2007, pp. 107–115., doi:10.1007/s10916-007-9113-9.

\bibitem{Wang} Wang, Louis Zizhao, et al. “Availability and Variability in Guidelines on Diabetic Retinopathy Screening in Asian Countries.” British Journal of Ophthalmology, vol. 101, no. 10, 2017, pp. 1352–1360., doi:10.1136/bjophthalmol-2016-310002.

\bibitem{Yao} Yao, Xin. “Evolving Artificial Neural Networks.” Proceedings of the IEEE, vol. 87, no. 9, 1999, pp. 1423–1447., doi:10.1109/5.784219.

\bibitem{Szegedy} Szegedy, Christian, et al. “Rethinking the Inception Architecture for Computer Vision.” 2016 IEEE Conference on Computer Vision and Pattern Recognition (CVPR), 2016, doi:10.1109/cvpr.2016.308.


\bibitem{Seth} Seth ,S et al “A hybrid deep learning model for detecting diabetic retinopathy”, In Journal of Statistics and Management Systems, Taylor Francis, 21 (4), pages: 569–574 2018

\bibitem{Rakhlin} Rakhlin, A. (2018, January 1). “Diabetic Retinopathy detection through integration of Deep Learning classification framework.” Retrieved from https://doi.org/10.1101/225508

\bibitem{Pedersen} Pedersen, S. J. K.  "Circular Hough Transform," Aalborg University, Vision, Graphics and Interactive Systems,November 2007. 


\bibitem{Chen} Chen, Xiang et al, "A novel method for automatic Hard Exudates detection in color retinal images", 2012 International Conference on Machine Learning and Cybernetics, pp. 1175-1181, 2012.

\bibitem{Decencière} Decencière, Etienne, et al. “Feedback On A Publicly Distributed Image Database: The Messidor Database.” Image Analysis and Stereology, vol. 33, no. 3, 2014, p. 231., doi:10.5566/ias.1155.

\bibitem{svm} Support-vector machine. (2019, September 24). Retrieved from https://en.wikipedia.org/wiki/Support-vector\textunderscore machine
\bibitem{BN} “Batch Normalization.” Wikipedia, Wikimedia Foundation, 27 July 2019, https://en.wikipedia.org/wiki/Batch\textunderscore normalization.

\bibitem{Acharya2} Rajendra Acharya, et al. “Application of Higher Order Spectra for the Identification of Diabetes Retinopathy Stages.” Journal of Medical Systems, vol. 32, no. 6, 2008, pp. 481–488., doi:10.1007/s10916-008-9154-8.

\bibitem{Kaggle} “Diabetic Retinopathy Detection.” Kaggle, https://www.kaggle.com/c/diabetic-retinopathy-detection/overview.

\bibitem{Priya}Priya, R et al “SVM and Neural Network Based Diagnosis of Diabetic Retinopathy.” International Journal of Computer Applications, vol. 41, no. 1, 2012, pp. 6–12., doi:10.5120/5503-7503.

\bibitem{Szeliski} Szeliski, Richard. Computer Vision: Algorithms and Applications. Springer, 2011.

\bibitem{Changing} “Changing the Contrast and Brightness of an Image!” OpenCV, https://docs.opencv.org/3.4/d3/dc1/tutorial\textunderscore basic\textunderscore linear\textunderscore transform.html.

\bibitem{HSV} “HSL and HSV.” Wikipedia, Wikimedia Foundation, 23 Sept. 2019, https://en.wikipedia.org/wiki/HSL\textunderscore and \textunderscore HSV.

\bibitem{Ramprasaath} Ramprasaath R., et al. “Grad-CAM: Visual Explanations from Deep Networks via Gradient-Based Localization.” 2017 IEEE International Conference on Computer Vision (ICCV), 2017, doi:10.1109/iccv.2017.74.

\bibitem{ROC} “Classification: ROC Curve and AUC  |  Machine Learning Crash Course.” Google, Google, https://developers.google.com/machine-learning/crash-course/classification/roc-and-auc.

\bibitem{clahe} Pizer, Amburn, Austin, et al. "Adaptive Histogram Equalization and Its Variations." Computer Vision, Graphics, and Image Processing 39 (1987) 355-368.


\end{thebibliography}
