\pagestyle{plain}

\chapter*{Περίληψη}
%\rule{\textwidth}{2pt}\\
Στην παρούσα διπλωματική εργασία γίνεται χρήση βαθιών συλλεκτικών δικτύων για την ανίχνευση της χρήζουσας θεραπείας Διαβητικής Αμφιβληστροειδοπάθειας(rDR). Το νευρωνικό που επιλέχθηκε είναι το Inception V3. Κατά την εκπαίδευση του δικτύου, τα δεδομένα αυξάνονται με αλλαγή της θέσης των εικονοστοιχείων και της τιμής τους. Τα παραπάνω επιτυγχάνονται με περιστροφή των εικόνων και προσαρμογή της αντίθεσης, της έντασης, του κορεσμού και της απόχρωσης των εικονοστοιχείων. 
Επίσης, εκπαιδεύονται 9 μοντέλα και πραγματοποιούνται οι μέθοδοι Ensemble με Ψηφοφορία και Ensemble Μέσης Τιμής. 

Επιπλέον, γίνεται χρήση του ταξινομητή Support Vector Machine(SVM) με γραμμική και Radial basis συνάρτηση πυρήνα. Για την ταξινόμηση των εικόνων στις κλάσεις ο ταξινομητής SVM δέχεται ως είσοδο, την έξοδο του προτελευταίου πλήρες συνδεδεμένου επιπέδου του εκπαιδευμένου νευρωνικού. Τέλος, έγινε οπτικοποίηση των περιοχών στις οποίες, σύμφωνα με το νευρωνικό, εντοπίζεται η ασθένεια.

Τα καλύτερα αποτελέσματα δόθηκαν με τη μέθοδο Ensemble Μέσης Τιμής με 9 μοντέλα ωστόσο όλα τα υπόλοιπα πειράματα που διεξήχθησαν και θα παρουσιαστούν στη συνέχεια, έδωσαν παραπλήσια αποτελέσματα. Το σύστημα αυτό, πετυχαίνει 0.91 AUC στο σύνολο δεδομένων Kaggle και 0.94 AUC στο σύνολο δεδομένων Messidor 2. Για το σημείο υψηλής ευαισθησίας, οι μετρικές ευαισθησία και εξειδίκευση είναι 0.9 και 0.69 αντίστοιχα για το σύνολο δεδομένων Kaggle και 0.9 και 0.79 για το σύνολο δεδομένων Messidor 2. Για το σημείο υψηλής εξειδίκευσης, οι μετρικές ευαισθησία και εξειδίκευση  είναι 0.76 και 0.92 αντίστοιχα για το σύνολο δεδομένων Kaggle και 0.8 και 95 για το σύνολο δεδομένων Messidor 2.
