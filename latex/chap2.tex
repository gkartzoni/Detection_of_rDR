\chapter{Βιβλιογραφική Επισκόπηση }
\label{chap:2}
\thispagestyle{plain}


\section{Υπάρχουσες Προσεγγίσεις}
\label{sec:2.1}
Στη βιβλιογραφία απαντώνται δύο προσεγγίσεις για την αυτόματα ανίχνευση της Διαβητικής Αμφιβληστροειδοπάθειας. Στην πρώτη προσέγγιση, γίνεται εξαγωγή συγκεκριμένων χαρακτηριστικών της ασθένειας(χωρίς τη χρήση νευρωνικών δικτύων). Τα χαρακτηριστικά που εξάγονται συνήθως είναι: σκληρά εξιδρώματα, μικροανευρίσματα, νεοαγγεία και βαμβακόμορφες κηλίδες. Επιπλέον εξάγονται κάποια χαρακτηριστικά της εικόνας πχ αντίθεση και διακύμανση και κάποια στοιχεία της δομής του αμφιβληστροειδούς πχ αγγεία, οπτικός δίσκος κα. Συνήθως σε αυτή τη προσέγγιση αρχικά γίνεται επεξεργασία των εικόνων ώστε να επιτευχθεί ενίσχυση των χαρακτηριστικών τους και στη συνέχεια εφαρμόζονται διάφορες τεχνικές για την εξαγωγή των εν λόγω χαρακτηριστικών\cite{Nayak}. Για την τελική πρόβλεψη εφαρμόζονται πληθώρα αλγορίθμων όπως μηχανές διανυσμάτων υποστήριξης(SVM), κ - κοντινότεροι γείτονες, τεχνητά νευρωνικά δίκτυα, δέντρα απόφασης κα. Επιπλέον, η τελική πρόβλεψη μπορεί να δοθεί βάση κάποιων κριτηρίων μεγέθους και αριθμού των χαρακτηριστικών της ασθένειας που ανιχνεύθηκαν. Οι κλάσεις ταξινόμησης μπορεί να είναι: έχει/δεν έχει rDr, έχει/δεν έχει DR ή ποιο βαθμό DR έχει(multi-class classification). Σε κάποιες εργασίες εξάγονται μόνο τα χαρακτηριστικά της ασθένειας, παραλείποντας τη διαδικασία ανίχνευσης της. Ωστόσο και αυτές οι προσεγγίσεις παίζουν σημαντικό ρόλο στη λύση του προβλήματος καθώς η κύρια δυσκολία της πρώτης προσέγγισης είναι η εξαγωγή των χαρακτηριστικών και όχι τόσο η επιλογή του τελικού ταξινομητή.

 
 Στη δεύτερη προσέγγιση γίνεται χρήση βαθιών συνελικτικών νευρωνικών δικτύων τα οποία δέχονται ως είσοδο ένα μεγάλο σύνολο εικόνων και την αντίστοιχη κλάση στην οποία ανήκουν. Το δίκτυο μαθαίνει τα χαρακτηριστικά της ασθένειας από το συνδυασμό εικόνας και της αντίστοιχης κλάσης. Το νευρωνικό δίκτυο θα μπορούσε να περιγραφεί ως μία συνάρτηση με πολλές παραμέτρους, οι οποίες προσαρμόζονται έτσι ώστε να προβλέπουν την κλάση που ανήκει κάθε εικόνα.  Επιπλέον, απαντώνται μετασχηματισμοί της εικόνας για την ενίσχυση των χαρακτηριστικών της, ενώ συνήθως γίνεται χρήση της τεχνικής αύξησης δεδομένων για την αποφυγή της υπερεκπαίδευσης και την αύξηση της επίδοσης του μοντέλου. Και σε αυτή την προσέγγιση οι κλάσεις ταξινόμησης μπορεί να είναι: έχει/δεν έχει rDr, έχει/δεν έχει DR ή ποιο βαθμό DR έχει(multi-class classification).

Οι διαφορές των δύο προσεγγίσεων δεν περιορίζονται μόνο στον τρόπο εξαγωγής των χαρακτηριστικών. Μία αρκετά σημαντική διαφορά είναι ο αριθμός των εικόνων που απαιτείται στην κάθε προσέγγιση, με την πρώτη να απαιτεί μερικές εκατοντάδες εικόνες ενώ η δεύτερη μερικές δεκάδες χιλιάδες εικόνες. Όπως είναι γνωστό τα συνελικτικά νευρωνικά δίκτυα απαιτούν αρκετές εικόνες ώστε να εκπαιδευτούν και να παρουσιάσουν καλά αποτελέσματα ενώ ο μεγάλος αριθμός εικόνων μειώνει την πιθανότητα για υπερεκπαίδευση. Έτσι, η δεύτερη προσέγγιση με συνελικτικα νευρωνικά δίκτυα απαιτεί την εύρεση ενός μεγάλου συνόλου δεδομένων ενώ συγχρόνως πολύ σημαντική είναι και η σωστή ανάθεση των εικόνων στις κλάσεις. 
Επιπλέον, ανάλογα με την κάμερα λήψης, τις συνθήκες φωτισμού και τη φυλή των εξεταζόμενων παρουσιάζονται μεγάλες διαφορές στις εικόνες του αμφιβληστροειδούς. Συνήθως στη πρώτη προσέγγιση παρατηρούνται πολύ καλά αποτελέσματα στο σύνολο δεδομένων που βασίστηκε ο αλγόριθμος ωστόσο δεν συμβαίνει το ίδιο για άλλα σύνολα δεδομένων. Αντίθετα,  τα  συνελικτικά νευρωνικά δίκτυα μπορούν να εκπαιδευτούν κατά τέτοιο τρόπο ώστε να γενικεύονται σε διάφορα σύνολα δεδομένων. Παρακάτω, παρουσιάζονται κάποιες σχετικές εργασίες με την ανίχνευση της ασθένειας.

\section{Παραδείγματα Βιβλιογραφίας}
\label{sec:2.2}

Στο Nayak et al\cite{Nayak} γίνεται χρήση τεχνικών επεξεργασίας εικόνας όπως προσαρμοσμένη εξισορρόπηση ιστογράμματος(adaptive histogram equalization), μορφολογικές τεχνικές για την ανίχνευση αιμοφόρων αγγείων και σκληρών εξιδρωμάτων και μέθοδοι ανάλυσης υφής για την μέτρηση της διακύμανσης της έντασης της εικόνας(αντίθεση). Αφού υπολογιστούν τα παραπάνω, δίνονται ως είσοδο σε ένα τεχνητό νευρωνικό δίκτυο. Συγκεκριμένα το νευρωνικό δέχεται ως είσοδο  το εμβαδόν και τη περίμετρο των αιμοφόρων αγγείων, το εμβαδόν των εξιδρωμάτων και το μέτρο της αντίθεσης ενώ η έξοδος  προβλέπει τρεις κλάσεις: norm, non proliferative DR, proliferative DR. Η μέθοδος πετυχαίνει ακρίβεια 93\%, ευαισθησία 90\% και εξειδίκευση 100\%. Για τη μέθοδο αυτή χρησιμοποιήθηκαν 140 δείγματα.



Στο Acharya et al\cite{Acharya2} χρησιμοποιούνται μη γραμμικά χαρακτηριστικά Higher-order spectra(HOS) ως είσοδο σε ταξινομητή SVM που πραγματοποιεί ταξινόμηση πέντε κλάσεων normal, mild, moderate, severe και proliferative DR. Συγκεκριμένα, στις εικόνες εφαρμόζεται η τεχνική εξισορρόπησης ιστογράμματος(Histogram equalization) για ενίσχυση της αντίθεσης, έπειτα οι εικόνες μετατρέπονται σε κλίμακα του γκρι και μετασχηματίζονται σε 1D δεδομένα με το μετασχηματισμό Radon. Τέλος, υπολογίζονται τα HOS χαρακτηριστικά και δίνονται ως είσοδο στον ταξινομητή. Για τη μέθοδο αυτή χρησιμοποιήθηκαν 300 δείγματα. Η μέθοδος πετυχαίνει ευαισθησία 82\% και εξειδίκευση 88\%.


Στο Priya et al \cite{Priya} συγκρίνεται η χρήση Πιθανοτικών νευρωνικών δικτύων(Probabilistic Neural network - PNN) και η χρήση Μηχανών διανυσμάτων υποστήριξης (SVM) για την τελική ταξινόμηση της ασθένειας. Όπως διαπιστώθηκε η χρήση SVM ως ταξινομητή υπερέχει έναντι του μοντέλου PNN, με το πρώτο να πετυχαίνει 97\% ακρίβεια ενώ το δεύτερο 89\%. 


Στο Gulshan et al \cite{Gulshan} στο οποίο βασίστηκε και η παρούσα διπλωματική χρησιμοποιήθηκε το νευρωνικό Inception V3. Έγινε χρήση ενός ιδιωτικού συνόλου δεδομένων με 128175 εικόνες, κάθε εικόνα  αξιολογήθηκε 3 με 7 φορές από ειδικούς οφθαλμίατρους για την αντιστοίχιση τους στις κλάσεις no, mild, moderate, severe, και proliferative DR και την ύπαρξη ή όχι οιδήματος της ωχράς κηλίδας. Επιπλέον, έγινε αύξηση των δεδομένων κατά την εκπαίδευση με αλλαγή της αντίθεσης, της φωτεινότητας, της απόχρωσης και του κορεσμού. Τέλος, υλοποιήθηκε ensemble 10 μοντέλων, με την τελική απόφαση να λαμβάνεται ως  ένας γραμμικός μέσος όρος των προβλέψεων. Επιπροσθέτως, έγιναν πειράματα με διαφορετικό αριθμό δειγμάτων εκπαίδευσης και αποδείχθηκε ότι μετά τα 55000 περίπου δείγματα εκπαίδευσης, με σταθερό σύνολο επικύρωσης 24360 εικόνες, δεν παρατηρείται βελτίωση της απόδοσης του μοντέλου. Η υλοποίηση πετυχαίνει άκρως εντυπωσιακά αποτελέσματα με AUC 0.991 για το ιδιωτικό σύνολο δεδομένων και 0.99 για το σύνολο δεδομένων Messidor 2.

Στο  Gargeya et al \cite{Gargeya} γίνεται χρήση ενός νευρωνικού με Residual αρχιτεκτονική. Η συγκεκριμένη αρχιτεκτονική περιγράφεται από την εξίσωση \ref{eq:resi}

\begin{equation} \label{eq:resi}
x_{l} = conv_{l}(x_{l-1}) + x_{l-1}
\end{equation}

όπου το $conv_{l}$ αναπαριστά το συνελικτικό επίπεδο $l$ και η έξοδος $x_{l}$ δίνεται ως ένα άθροισμα της εξόδου του συνελικτικού επιπέδου $l$ δηλαδή $conv_{l}(x_{l-1})$ και της εξόδου του προηγούμενου συνελικτικού επιπέδου δηλαδή $x_{l-1}$
Για την εκπαίδευση χρησιμοποιήθηκε ένα σύνολο δεδομένων με 75137, το οποίο χωρίστηκε στα σύνολα εκπαίδευσης, επικύρωσης και ελέγχου και οι εικόνες αξιολογήθηκαν από ειδικούς ώστε να αποδοθεί σε κάθε εικόνα η αντίστοιχη κλάση στην οποία ανήκει. Επίσης, χρησιμοποιήθηκε το Messidor 2 ως ένα επιπλέον σύνολο ελέγχου. Επιπροσθέτως, έγινε αύξηση δεδομένων με χρήση περιστροφής και προσαρμογής φωτεινότητας και αντίθεσης. Το μοντέλο πετυχαίνει 0.97 AUC στο σύνολο ελέγχου του βασικού συνόλου δεδομένων  και 0.94 στο Messidor 2. Στη συγκεκριμένη προσέγγιση ανιχνεύεται η DR, όχι η referable DR.


Στο Lin et al\cite{Lin} προτείνεται η χρήση εικόνων εντροπίας ως είσοδο στο νευρωνικό αφού πρώτα μετατραπούν σε εικόνες κλίμακας του γκρι ενώ στο Seth et al \cite{Seth} προτείνεται η χρήση συλλεκτικών νευρωνικών δικτύων με ταξινομητή SVM με γραμμικό πυρήνα. Ως είσοδος στο μοντέλο SVM δίνεται η έξοδος του προτελευταίου πλήρες συνδεδεμένου επιπέδου του νευρωνικού με 1024 νευρώνες.

